% https://www.youtube.com/watch?v=gjifrE_x4kw&feature=youtu.be

\title{\href{http://www.blender.org}{Blender}}

\maketitle
\tableofcontents

\chapter{\href{http://en.wikipedia.org/wiki/Blender_(software)}{Introduction}}

\section{What?}
\begin{itemize}
\item A GPL 3D computer graphics software for creating animated films,
  visual effects, art, 3D (printed) models, 3D applications and video
  games.
\end{itemize}

\section{Who and when?}
\begin{itemize}
\item Ton Roosendaal, the main contributor of Blender, founded NaN
  (Not a Number) Technologies in June 1998 and start sharing Blender
  as shareware.
\item NaN went backrupt in 2002, and together with Neo Geo,
open-source Blender using 
%\euro{}
$100,000$ Euros collected from the
community.
\end{itemize}

\section{Documentation}
\begin{enumerate}
\item The \href{http://www.blender.org}{Blender's website}.
\item The \href{http://wiki.blender.org}{Blender Wiki}.
\item \href{http://en.wikibooks.org/wiki/Blender_3D:_Noob_to_Pro}{Blender 3D: Noob to Pro}.
\end{enumerate}

\section{Projects}
\begin{enumerate}
\item \href{https://orange.blender.org}{Elephants Dream}.
\item \href{https://peach.blender.org}{Big Buck Bunny}.
\item \href{http://www.yofrankie.org}{Yo Frankie!}.
\item \href{https://durian.blender.org}{Sintel}.
\item \href{https://mango.blender.org}{Tears of Steel}.
\item \href{http://gooseberry.blender.org/cosmos-laundromat}{Cosmos Laundromat}.
\end{enumerate}

\chapter{An example:
\href{https://www.youtube.com/watch?v=y__uzGKmxt8}{The ceramic cup}}

\begin{verbatim}
/* Clear the project. */

[File] /* [] means clickable */
 +- [New]
     +- [Reload Start-Up File]

[Blender Render]
 +- [Cycles Render]

/* Some usefull information:

How to select an object: <Right mouse> click on it.

How to rotate the scene: Press and hold the <Middle mouse> button and move the mouse.

How to pan: Press and hold the <Shift> key, press and hold the <Middle mouse> button and move the mouse.

How to zoom: use the <Scrool wheel> of the mouse.

*/

/* Delete the cube. */

Select the cube with the <Right mouse> button.

Push the <X> key (delete) in the keyboard.

Click on ``Delete''.

/* Create the cup. */

/* Create a cylinder. */

[Add]
 +- [Mesh]
     +- [Cylinder]

(Add Cylinder) /* () means that this is a label */
 +- (Cap Fill Type)
     +- [Triangle Fan]

/* Change to Edit Mode. */

[Object Mode]
 +- [Edit Mode]

[Viewport Shading]
 +- [Wireframe]

/* Delete the top portion of the cylinder. */

Select the top center vertice of the cylinder using the <Right mouse> button.

Press the <delete> key in the keyboard.

Click on [Vertices].

/* Change the view to the front view (number <1> in the number pad) or use: */

[View]
 +- [Front]

/* Change to view mode to orthogonal mode (or perspective move). Number <5> or: */

[View]
 +- [View Persp/Ortho]

/* Zoom in */

Use the <Scrool mouse> wheel.

/* Sub-divide horizontally the object. */

Move the cursor over the object.

Push <Ctrl> + <R>.

Use the <Scrool mouse> wheel to have four sub-division lines.

Push the <Left mouse> button.

/* Deselect everything. */

Push the <A> key.

/* Select the top vertices of the cup. */

Press and release the <B> key.

/* Select the top vertices using a selection box. */

Click and hold the <Left mouse> button.
Moving the mouse.
Release the <Left mouse> button.

/* Move the top vertices of the cup down. */

Click on the top blue arrow and move it down to reduce the height of
the cup.

/* Scale a 90% the top vertices of the cup. */

Press the <S> key.

Type <.9> and press <Enter>.

Press the <A> key.

/* Select the top-center vertices of the cup. */

Press the <B> key.

Press the <S> key.

Type <1.05> and press <Enter>.

Press <A>.

/* Select the bottom vertices of the cup. */

Press <B>.

/* Select the bottom vertices using a selection box. */

Click and hold the <Left mouse> button.
Move the mouse.
Release the <Left mouse> button.

Press <S>.

Type <.8> and press <Enter>.

Press <A>.

/* Move the cup to the left of the window. */

Hold down the <Shift> key and hold the <Mittle mouse> button.

/* Make the handle of the cup. */

[Edit Mode]
 +- [Object Mode]

[Viewport Shading]
 +- [Solid]

/* Create a new origin point. */

Click on the <Left mouse> button on the free section of the window.

/* Create a path with the shape of the handle. */

[Add]
 +- [Curve]
     +- [Path]

/* Enter the Edit Mode. */

[Object Mode]
 +- [Edit Mode]

/* Shape of the handle. */

Click on the left node and move it up moving the arrows. 

/* Enter the Object Mode. */

[Edit Mode]
 +- [Object Mode]

/* Create the section of the handle. */

[Add]
 +- [Curve]
     +- [Circle]

/* Rotate the view. */

Press the <middle mouse button> and drag the mouse.

/* Scale the cicle. */

Press <S>.

Type <.2> and press <Enter>.

/* Use the circle to control the shape along the path. */

Select the handle by <Right mouse> clicking on it.

Click on the [Object Data] button.

(Geometry)
 +- (Bevel Object:)
     +- [BezierCircle]

/* Change the circle resolution (convert it in a square). */

<Right mouse> click on the circle.

(Shape)
 +- (Resolution:)
     +- [Preview U:] <- 1

/* Convert the handle into a mesh object. */

<Right mouse> click on the handle to select it.

Press <Alt> + <C>.

(Convert to)
 +- [Mesh from Curve/Meta/Suft/Text]

/* Delete the (square) circle. */

<Right mouse> click on the circle.

Press the <Delete> key.

(OK?)
 +- [Delete X]

/* Check if the handle is lined up properly with the cup. */

Press the number <7> on the number pad to change the top view.

Press the <Shift>.

Press and hold the <Middle mouse> button while your drag the two pieces
towards the center of the window.

/* Switch the from view. */ 

Press the number <1> on the number pad. 

/* Move the cup a little bit closer to the handle. */

<Right mouse> click on the cup to select it.

Grab the red arrow in the direction of the haldle.

/* Join the cup and the haldle together. */

Select them all by holding down the <Shift> key while <Mouse right> clicking
on them.

Press <Ctrl> + <J>.

/* Change to Edit Mode. */

[Object Mode]
 +- [Edit Mode]

/* Connect the haldle and the cup. */

Press <A> to deselect everything.

[Face select - Shift-Click for multiple modes, Crtl-Click expand selection]

/* Rotate the objects. */

Hold down the <Middle mouse> button.

/* Delete some faces of the cup. */

<Right mouse> click on the faces that will touch the hanndle.

Use the <Shift> key to add selections.

Click the <Delete> key.

(Delete)
 +- [Faces]

/* Join the handle to the cup. */

[Edge select -Shift-Click for multiple modes, Ctrl-Click expands/contracts selections.]

Holding down the <Shift> key, <Right mouse> click in the edges of the cup
and the handle we want to join.

Press <F> to create the new joining face.

Repeat the process for the rest of the edges.

Press <A>.

/* Save the creation. */

[File]
 +- [Save as]

Input the name of your file.

[Save As Blender File]

/* Starts the second video. */

/* Make the inside of the cup. */

[Viewport Shading]
 +- [Wireframe]

Press <1> in the number pad to switch to front view.

[Vertex select - Shift-Click for multiple modes, Ctrl-Click contracts selection]

/* Select the top vertices of the cup. */

Press <B>.

Holding down the <Left mouse> button, drag the selection box around the
vertices.

/* Extrude the selected vertices ... */

Press <E>.

/* ... to the same point. */

Press <Enter>.

/* Rotate the cup .*/

/* Scale 80%. */

Press <S>.

Type <.8> and <Enter>.

/* Switch the front view. */

Press <1>.

/* Extrude again the top vertices. */

Press <E>.

/* Move the mouse to see better what we are doing. */

/* Restrict the movement to be straight up and down. */

Press <Z>.

/* Move the new vertices to the same height than outer ones. */

Click the <Left mouse> button.

/* Scale. */

Press <S>.

/* Move the mouse until place the new faces parallel with the outside
of the cup. */

Press the <Left mouse> button.

/* Extrude again. */

Press <E>.

/* Limit movement. */

Press <Z>.

/* Move the mouse until place the new vertices to the same height than
the outer ones. */

/* Scale. */

Press <S>.

/* Move the mouse until place the new faces parallel with the outside
of the cup. */

Press the <Left mouse> button.

/* Extrude. */

Press <E>.

/* Limit. */

Press <Z>.

/* Move the mouse until place the new vertices to the same height than
the outer ones. */

Press the <Left mouse> button.

/* Scale. */

Press <S>.

/* Move the mouse until place the new faces parallel with the outside
of the cup. */

Press the <Left mouse> button.

/* Extrude ... */

Press <E>.

/* ... to the same point. */

Press <Enter>.

/* Change the view going up. */

/* Merge all the bottom vertices to the center of the cup. */

(Mesh Tools)
 +- (Remove:)
     +- [Merge]
       +- [At Center]

/* Switch to solid. */

[Viewport Shading]
 +- [Solid]

/* Switch to Object Mode. */

[Edit Mode]
 +- [Object Mode]

/* Smooth the edges. */

(Edit)
 +- (Shading:)
     +- [Smooth]

[Object Modifiers]
 +- [Add Modifier]
     +- [Subdivision Surface]

[Subdivisions:]
 +- [View:] <- 3
 +- [Render:] <- 3

/* Make it look like it's made of Chinese ceramic material. */

/* Be sure [Cycles Render] has been selected. */

[Material]
 +- [New]

[Surface]
  +- [Difuse BSDF]
      +- [Mix Shader]

[Shader:]
 +- [Diffuse BSDF]
 +- [Color:] <- Light gray color.

[Shader:]
 +- [Glossy BSDF]
 +- [Roughne]: 0

Type <0> and press <Enter>.

[Surface]
 +- [Fac:] <- 0.14

Type <.14> and press <Enter>.

/* Let's make the table. */

/* Change the view to the front one. */

Press <1>.

/* Put the origin point below the cup. */

<Right mouse> click below de cup.

[Add]
 +- [Mesh]
     +- [Plane]

/* Scale the plane up to 3 times its current size. */

Press <S>.

Type <3> and press <Enter>.

/* Switch the front view. */

Press <1>.

/* Move up the plane until the bottom of the cup .*/

Click on the blue arrow using the <Right mouse> botton and move the mouse.

/* Set the material for this plane. */

[Material]
 +- [New]
     +- [Color:] <- Light blue color.

/* Add a light source. */

/* Zoom out until see the camera. */

/* Put the origin point next to the camera. */

[Add]
 +- [Mesh]
     +- [Plane]

/* Rotate the new plane 45 degrees. */

Press <R>.

Type <.45> and press <Enter>.

/* Scale the new plane up to be five times its current size. */

Press <S>.

Type <5> and press <Enter>.

/* Allow the plane to emit light. */

[New]
 +- [Surface]
     +- [Surface:]
         +- [Emission]

/* Set the amount of light emitted to 10. */

[Strength:] <- 10

/* Switch to the camera view. */

Press <0> on the number pad.

/* To center the scene from the camera view. */

[View]
 +- [Properties]

[Lock Camera to View]

[View]
 +- [Properties]

Use the <Scroll mouse> wheel to zoom.

<Shift> + hold the <Middle mouse> button to pan the view.

/* Render with a low number of samples. */

[Render (camera)]
 +- [Render (button)]

/* Render with a medium number of samples. */

[Render (camera)]
 +- [Render (button)]

[Sampling]
 +- Samples:
    +- [Render:] <- 250

/* Prevent "fireflies" (unwanted white pixels). */

[Sampling]
 +- [Clamp:] <- .98

[Render (button)]

/* Wait! */

/* Save the image. */

Press <F3>.

Select a ".pgm" image.

/* Return to the previous view. */

Press <Esc>.

/* Return to the render view. */

Press <F11>.

/* Save the project. */

[File]
 +- [Save]

\end{verbatim}

